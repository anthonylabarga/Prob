\documentclass[12pt]{article}
\usepackage[utf8]{inputenc}
\usepackage{fullpage,amsmath,amsfonts,amssymb,graphicx,bbm}

\begin{document}

\begin{flushright}

Anthony Labarga\\
Due 9/10
\end{flushright}

\begin{center}
Homework \#1
\end{center}

\begin{enumerate}

\item

\begin{itemize}

\item[a)] To find $\limsup$ of $(-1)^n(4+\frac{1}{n})$, we need $\overline{s}$ s.t. $s_n<\overline{s}+\epsilon(>0)$ for large $n$ and $s_n>\overline{s}-\epsilon(>0)$ for infinitely many $n$. In this case, $s_n$ alternates between slightly more than 4 and slightly less than -4. So for really large $n$, the $\sup$ is approach 4, and we can always find a small $\epsilon > \frac{1}{n}$ s.t. $s_n<4+\epsilon$ and $s_n>4-\epsilon$ from that $n$ on. So $\overline{s}=4$.

Similarly, to find $\liminf$, we need $\underline{s}$ s.t. $s_n>\underline{s}-\epsilon$ for large $n$ and $s_n<\underline{s}+\epsilon$ for infinitely many $n$. As $n$ gets large, we can always find a small $\epsilon > \frac{1}{n}$ s.t. $s_n>-4-\epsilon$, and for that $\epsilon$, $s_n<-4+\epsilon$ from that $n$ on. 

\item[b)] To find $\limsup$ of $\sin(\frac{n\pi}{2}+e^{-n})$, notice that $\sin(\frac{n\pi}{2})$ oscillates as $\{1,0,-1,0,1,\dots\}$. $e^{-n}$ is a small positive number, so this sequence oscillates between slightly less than 1 and slightly more than -1. Since the $e^{-n}$ term decays to 0, the sup monotonically approaches 1 and the inf monotonically approaches -1. Thus we can always find an $\epsilon$ s.t. $s_n<1+\epsilon$ and $s_n>1-\epsilon$ from that $n$ on, as well as $s_n>-1-\epsilon$ and $s_n<-1+\epsilon$ from that $n$ on. So $\overline{s}=1$ and $\underline{s}=-1$.

\end{itemize}

\item If $A_n= \left( -\frac{1}{n},5+\frac{(-1)^n}{n} \right)$, then 
$$\bigcup^\infty_{k=n} A_k = \begin{cases} \left( -\frac{1}{n},5+\frac{1}{n} \right) & \mbox{ if n is even}\\ \left( -\frac{1}{n},5+\frac{1}{n+1} \right) & \mbox{ if n is odd} \end{cases}$$

So $A_n \mbox{ io}=\overline{\lim} A_n = \bigcap^\infty_{n=1} \bigcup_{k=n}^\infty A_k = [0,5]$ (since all of these ever-constricting sets include both 0 and 5).

Likewise, we have $\bigcap^\infty_{k=n} A_k = [0,5)$, since on the left we are constricting to 0, but on the right we are oscillating from just less than 5 to just more than 5. Thus the limiting right endpoint is exclusive of 5. $\bigcup^\infty_{n=1}[0,5)=\underline{\lim} A_n= A_n \mbox{ ev}=[0,5)$.

\item Given $A_i= \begin{cases} B & \mbox{i is even} \\ C & \mbox{i is odd} \end{cases}$, $\bigcup^\infty_{k=n} A_k = \dots \cup B \cup C \cup B \cup C \dots = B \cup C$ and $\bigcap^\infty_{k=n} A_k = \dots \cap B \cap C \cap B \cap C \dots = B \cap C$. So 

$$ A_n \mbox{ io}=\bigcap^\infty_{n=1}\bigcup^\infty_{k=n} A_k = (B \cup C) \cap (B \cup C)\dots = (B \cup C)$$

$$ A_n \mbox{ ev}=\bigcup^\infty_{n=1}\bigcap^\infty_{k=n} A_k = (B \cap C) \cup (B \cap C)\dots = (B \cap C)$$

\item To show that $\mathbbm{1}_{A_n\mbox{ io}}(\omega)=\overline{\lim}_{n\to\infty}\mathbbm{1}_{A_n}(\omega)$ $\forall \omega$, note that 

$$\overline{\lim_{n\to\infty}}\mathbbm{1}_{A_n}(\omega)=\lim_{k \to \infty} \sup_{n \geq k} \mathbbm{1}_{A_n}(\omega)=\lim_{k \to \infty} \max \{\mathbbm{1}_{A_n}(\omega)\}=\lim_{k \to \infty} \mathbbm{1}_{\bigcup^\infty_{n \geq k} A_k}(\omega)=\mathbbm{1}(\mathbbm{1}_{\bigcup^\infty_{k=1}}=1\forall k)$$ $$=\mathbbm{1}_{ \bigcap^\infty_{k=1} \bigcup^\infty_{n=k}A_k} = \mathbbm{1}_{A_n\mbox{ io}}(\omega) $$

To show that $\mathbbm{1}_{A_n\mbox{ ev}}(\omega)=\underline{\lim}_{n\to\infty}\mathbbm{1}_{A_n}(\omega)$ $\forall \omega$, note that 

$$\underline{\lim}_{n\to\infty}\mathbbm{1}_{A_n}(\omega)=\lim_{k \to \infty} \inf_{n \geq k} \mathbbm{1}_{A_n}(\omega)=\lim_{k \to \infty} \min \{\mathbbm{1}_{A_n}(\omega)\}=\lim_{k \to \infty} \mathbbm{1}_{\bigcap^\infty_{n \geq k} A_k}(\omega)$$ $$=\mathbbm{1}(\mathbbm{1}_{\bigcap^\infty_{k=1}}=1\forall k>k_0) = \mathbbm{1}_{A_n\mbox{ ev}}(\omega) $$

Lastly, we need to show that $A_n \to A \Leftrightarrow \lim_{n \to \infty} \mathbbm{1}_{A_n}(\omega)=\mathbbm{1}_{A}(\omega)$. $A_n \to A$ iff $A=A_n\mbox{ ev}=A_n\mbox{ io}$, so $\lim \mathbbm{1}_{A_n}(\omega)\lim \mathbbm{1}_{A_n\mbox{ io}}=\lim \mathbbm{1}_{A_n\mbox{ ev}}=\mathbbm{1}_{A}(\omega)$.

\item Claim: $(A_n \cup B_n)\mbox{ io} = (A_n \mbox{ io})\cup (B_n\mbox{ io})$

Proof: This proof relies on the distributivity of the union and intersection operators. 

$$(A_n \cup B_n)\mbox{ io} = \bigcap^\infty_{n=1}\bigcup^\infty_{k=n}(A_n \cup B_n) = \bigcap^\infty_{n=1} \left( \left( \bigcup^\infty_{k=n}A_n \right) \bigcup \left( \bigcup^\infty_{k=n}B_n \right) \right)$$

$$ = \left( \bigcap^\infty_{n=1}\bigcup^\infty_{k=n}A_n \right) \bigcup \left( \bigcap^\infty_{n=1} \bigcup^\infty_{k=n}B_n \right) = (A_n \mbox{ io})\cup (B_n\mbox{ io}) $$

\item Claim: $(A_n \cap B_n)\mbox{ ev} = (A_n \mbox{ ev})\cap (B_n\mbox{ ev})$

Proof: This proof relies on the distributivity of the union and intersection operators. 

$$(A_n \cap B_n)\mbox{ ev} = \bigcup^\infty_{n=1}\bigcap^\infty_{k=n}(A_n \cap B_n) = \bigcup^\infty_{n=1} \left( \left( \bigcap^\infty_{k=n}A_n \right) \bigcap \left( \bigcap^\infty_{k=n}B_n \right) \right)$$

$$ = \left( \bigcup^\infty_{n=1}\bigcap^\infty_{k=n}A_n \right) \bigcap \left( \bigcup^\infty_{n=1} \bigcap^\infty_{k=n}B_n \right) = (A_n \mbox{ ev})\cap (B_n\mbox{ ev}) $$

\item Claim: $A_n \to A$ and $B_n \to B$ implies $A_n \cup B_n \to A \cup B$.

Proof: This proof makes use of the result from problem 5.

$A_n \to A \Leftrightarrow A = A_n \mbox{ io} = A_n \mbox{ ev}$, and likewise with $B_n$. We can establish the limit of $(A_n \cup B_n)$ by finding its ev and io. In view of (5), $(A_n \cup B_n)\mbox{ io} = (A_n\mbox{ io}) \cup (B_n\mbox{ io})=A \cup B$, and $(A_n \cup B_n)\mbox{ ev} = (A_n\mbox{ ev}) \cup (B_n\mbox{ ev})=A \cup B$. Since the ev and io of $A_n \cup B_n$ coincide, $A_n \cup B_n \to A \cup B$.

\item $A_n \to A$ and $B_n \to B$ implies $A_n \cap B_n \to A \cap B$.

Proof: This proof makes use of the result from problem 6.

$A_n \to A \Leftrightarrow A = A_n \mbox{ io} = A_n \mbox{ ev}$, and likewise with $B_n$. We can establish the limit of $(A_n \cap B_n)$ by finding its ev and io. In view of (6), $(A_n \cap B_n)\mbox{ io} = (A_n\mbox{ io}) \cap (B_n\mbox{ io})=A \cap B$, and $(A_n \cap B_n)\mbox{ ev} = (A_n\mbox{ ev}) \cap (B_n\mbox{ ev})=A \cap B$. Since the ev and io of $A_n \cap B_n$ coincide, $A_n \cap B_n \to A \cap B$.

\item To express $(A_n \cap A_{n+1}^c)\mbox{ io}$ in terms of $A_n\mbox{ ev}$ and $A_n\mbox{ io}$, notice that $(A_n \cap A_{n+1}^c)\mbox{ io} = (A_n \mbox{ io}) \cap (A_{n+1}^c\mbox{ io}) = (A_n \mbox{ io}) \cap (A_{n+1}\mbox{ ev})^c$. Lastly, notice that $A_{n+1}\mbox{ ev} = A_{n}\mbox{ ev}$ by shifting the indices on the union and intersection operators, so this is equal to $(A_n \mbox{ io}) \cap (A_{n}\mbox{ ev})^c$.

\item This can be done by proving the bi-implication 

$$A \in \{\omega \in \Omega : g_n(\omega) \to g(\omega)\} \Leftrightarrow A \in \bigcap^{\infty}_{k=1} A_n^{(k)}\mbox{ ev}$$

Suppose $A \in \{\omega \in \Omega : g_n(\omega) \to g(\omega)\}$. For pointwise convergence to occur on $\Omega$, $\forall k>0$, there must be some $n_0$ such that for $n\geq n_0$, $A \in A_n^{(k)}$; i.e. $A\in  A_n^{(k)}\mbox{ ev}$ $\forall k$. This last can be reformulated as $A \in \bigcap^{\infty}_{k=1} A_n^{(k)}\mbox{ ev}$.

Suppose $A \in \bigcap^{\infty}_{k=1} A_n^{(k)}\mbox{ ev}$. Then $\forall k$, $A\in  A_n^{(k)}\mbox{ ev}$. This means that for every $k$, there must be some $n_0$ such that for $n\geq n_0$, $A \in A_n^{(k)}$; i.e. for $n \geq n_0$ we have $|g_n(\omega)-g(\omega)|<k^{-1}$. This implies that pointwise convergence is occurring on $\Omega$, $\forall k>0$. So $A \in \{\omega \in \Omega : g_n(\omega) \to g(\omega)\}$.

\end{enumerate}


\end{document}