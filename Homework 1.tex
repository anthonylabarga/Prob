\documentclass[12pt]{article}
\usepackage[utf8]{inputenc}
\usepackage{fullpage,amsmath,amsfonts,amssymb,graphicx}

\begin{document}

\begin{flushright}

Anthony Labarga\\
Due 9/10
\end{flushright}

\begin{center}
Homework \#1
\end{center}

\begin{enumerate}

\item

\begin{itemize}

\item[a)] To find $\limsup$ of $(-1)^n(4+\frac{1}{n})$, we need $\overline{s}$ s.t. $s_n<\overline{s}+\epsilon(>0)$ for large $n$ and $s_n>\overline{s}-\epsilon(>0)$ for infinitely many $n$. In this case, $s_n$ alternates between slightly more than 4 and slightly less than -4. So for really large $n$, the $\sup$ is approach 4, and we can always find a small $\epsilon > \frac{1}{n}$ s.t. $s_n<4+\epsilon$ and $s_n>4-\epsilon$ from that $n$ on. So $\overline{s}=4$.

Similarly, to find $\liminf$, we need $\underline{s}$ s.t. $s_n>\underline{s}-\epsilon$ for large $n$ and $s_n<\underline{s}+\epsilon$ for infinitely many $n$. As $n$ gets large, we can always find a small $\epsilon > \frac{1}{n}$ s.t. $s_n>-4-\epsilon$, and for that $\epsilon$, $s_n<-4+\epsilon$ from that $n$ on. 

\item[b)] To find $\limsup$ of $\sin(\frac{n\pi}{2}+e^{-n})$, notice that $\sin(\frac{n\pi}{2})$ oscillates as $\{1,0,-1,0,1,\dots\}$. $e^{-n}$ is a small positive number, so this sequence oscillates between slightly less than 1 and slightly more than -1. Since the $e^{-n}$ term decays to 0, the sup monotonically approaches 1 and the inf monotonically approaches -1. Thus we can always find an $\epsilon$ s.t. $s_n<1+\epsilon$ and $s_n>1-\epsilon$ from that $n$ on, as well as $s_n>-1-\epsilon$ and $s_n<-1+\epsilon$ from that $n$ on. So $\overline{s}=1$ and $\underline{s}=-1$.

\end{itemize}

\item If $A_n= \left( -\frac{1}{n},5+\frac{(-1)^n}{n} \right)$, then 
$$\bigcup^\infty_{k=n} A_k = \begin{cases} \left( -\frac{1}{n},5+\frac{1}{n} \right) & \mbox{ if n is even}\\ \left( -\frac{1}{n},5+\frac{1}{n+1} \right) & \mbox{ if n is odd} \end{cases}$$

So $A_n \mbox{ io}=\overline{\lim} A_n = \bigcap^\infty_{n=1} \bigcup_{k=n}^\infty A_k = [0,5]$ (since all of these ever-constricting sets include both 0 and 5).

Likewise, we have $\bigcap^\infty_{k=n} A_k = [0,5)$, since on the left we are constricting to 0, but on the right we are oscillating from just less than 5 to just more than 5. Thus the limiting right endpoint is exclusive of 5. $\bigcup^\infty_{n=1}[0,5)=\underline{\lim} A_n= A_n \mbox{ ev}=[0,5)$.

\item Given $A_i= \begin{cases} B & \mbox{i is even} \\ C & \mbox{i is odd} \end{cases}$, $\bigcup^\infty_{k=n} A_k = \dots \cup B \cup C \cup B \cup C \dots = B \cup C$ and $\bigcap^\infty_{k=n} A_k = \dots \cap B \cap C \cap B \cap C \dots = B \cap C$. So 

$$ A_n \mbox{ io}=\bigcap^\infty_{n=1}\bigcup^\infty_{k=n} A_k = (B \cup C) \cap (B \cup C)\dots = (B \cup C)$$

$$ A_n \mbox{ ev}=\bigcup^\infty_{n=1}\bigcap^\infty_{k=n} A_k = (B \cap C) \cup (B \cap C)\dots = (B \cap C)$$

\item

\item Claim: $(A_n \cup B_n)\mbox{ io} = (A_n \mbox{ io})\cup (B_n\mbox{ io})$

Proof: This proof relies on the distributivity of the union and intersection operators. 

$$(A_n \cup B_n)\mbox{ io} = \bigcap^\infty_{n=1}\bigcup^\infty_{k=n}(A_n \cup B_n) = \bigcap^\infty_{n=1} \left( \left( \bigcup^\infty_{k=n}A_n \right) \bigcup \left( \bigcup^\infty_{k=n}B_n \right) \right)$$

$$ = \left( \bigcap^\infty_{n=1}\bigcup^\infty_{k=n}A_n \right) \bigcup \left( \bigcap^\infty_{n=1} \bigcup^\infty_{k=n}B_n \right) = (A_n \mbox{ io})\cup (B_n\mbox{ io}) $$

\item Claim: $(A_n \cap B_n)\mbox{ ev} = (A_n \mbox{ ev})\cap (B_n\mbox{ ev})$

Proof: This proof relies on the distributivity of the union and intersection operators. 

$$(A_n \cap B_n)\mbox{ ev} = \bigcup^\infty_{n=1}\bigcap^\infty_{k=n}(A_n \cap B_n) = \bigcup^\infty_{n=1} \left( \left( \bigcap^\infty_{k=n}A_n \right) \bigcap \left( \bigcap^\infty_{k=n}B_n \right) \right)$$

$$ = \left( \bigcup^\infty_{n=1}\bigcap^\infty_{k=n}A_n \right) \bigcap \left( \bigcup^\infty_{n=1} \bigcap^\infty_{k=n}B_n \right) = (A_n \mbox{ ev})\cap (B_n\mbox{ ev}) $$

\item Claim: $A_n \to A$ and $\mbox{ ev} = (A_n \mbox{ ev})\cap (B_n\mbox{ ev})$

\item

\item

\item

\end{enumerate}


\end{document}