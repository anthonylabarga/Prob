\documentclass[12pt]{article}
\usepackage[utf8]{inputenc}
\usepackage{amsmath,amsfonts,amssymb,graphicx}

\begin{document}

\begin{flushright}

Anthony Labarga\\
Due 9/10
\end{flushright}

\begin{center}
Homework \#1
\end{center}

\begin{enumerate}

\item

\begin{itemize}

\item[a)] To find $\limsup$ of $(-1)^n(4+\frac{1}{n})$, we need $\overline{s}$ s.t. $s_n<\overline{s}+\epsilon(>0)$ for large $n$ and $s_n>\overline{s}-\epsilon(>0)$ for infinitely many $n$. In this case, $s_n$ alternates between slightly more than 4 and slightly less than -4. So for really large $n$, the $\sup$ is approach 4, and we can always find a small $\epsilon > \frac{1}{n}$ s.t. $s_n<4+\epsilon$ and $s_n>4-\epsilon$ from that $n$ on. So $\overline{s}=4$.

Similarly, to find $\liminf$, we need $\underline{s}$ s.t. $s_n>\underline{s}-\epsilon$ for large $n$ and $s_n<\underline{s}+\epsilon$ for infinitely many $n$. As $n$ gets large, we can always find a small $\epsilon > \frac{1}{n}$ s.t. $s_n>-4-\epsilon$, and for that $\epsilon$, $s_n<-4+\epsilon$ from that $n$ on. 

\item[b)] To find $\limsup$ of $\sin(\frac{n\pi}{2}+e^{-n})$, notice that $\sin(\frac{n\pi}{2})$ oscillates as $\{1,0,-1,0,1,\dots\}$. $e^{-n}$ is a small positive number, so this sequence oscillates between slightly less than 1 and slightly more than -1. Since the $e^{-n}$ term decays to 0, the sup monotonically approaches 1 and the inf monotonically approaches -1. Thus we can always find an $\epsilon$ s.t. $s_n<1+\epsilon$ and $s_n>1-\epsilon$ from that $n$ on, as well as $s_n>-1-\epsilon$ and $s_n<-1+\epsilon$ from that $n$ on. So $\overline{s}=1$ and $\underline{s}=-1$.

\end{itemize}

\item If $A_n=(\frac{-1}{n},)$

\item

\item

\item 

\item

\item 

\item

\item

\item

\end{enumerate}


\end{document}